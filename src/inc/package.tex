%[1 Encodage du document
\usepackage[T1]{fontenc}
\usepackage{lmodern}
\usepackage[utf8]{inputenc}

%[1 Packages d'aide
\usepackage{xspace}
% permet de gérer automatiquement les espace à la fin des macros

\usepackage{lipsum}
% Pour avoir du faux texte pour la mise en page


%[1 Mise en page
\usepackage{multicol}
% pour la gestion de colones dans le texte

\usepackage[np]{numprint}
% Affiche les chiffres de manière jolie

% vim: set ts=3 sts=3 sw=3 tw=100 fdm=marker foldmarker=%[,%] filetype=tex:
