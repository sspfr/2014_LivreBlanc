\dedication{% dédicace {{4
	\begin{list}{}{%
		\leftmargin .4%
		\linewidth%
		\rightmargin -2cm%
	}
		\raggedleft%
		\small%
		\item
		\og La disposition du code a engendré des guerres plus nombreuses et plus sanglantes
			que n'importe quel autre aspect de la programmation.
			\medskip

			Quel est donc la meilleure pratique dans ce cas~?\\
			Devez-vous utiliser le styla classique de \textsc{Kernighan \& Ritchie}~?\\
			Ou opter pour le formatage du code à la BSD~?\\
			Ou adopter le principe de mise en page spécifié par le projet GNU~?\\
			Ou vous conformer aux principes de codage de Slashcode~?\\
			\medskip

			Bien-sur que non~!\\
			Chacun sait que \emph{[insérez ici votre style préféré]} est les seul véritable style de mise en forme,
					 le seul choix correct,
					 ainsi que l'a proclamé \emph{[insérez ici votre dieu de la programmation préféré]} depuis la nuit des temps~!
						 Tout autre choix est à l'évidence un choix absurde,
					 une hérésie délibérée et l'\oe uvre des Forces du Mal~!!!~»

					 \emph{Damian \textsc{Conway,\\ «~De l'art de programmer en Perl\fg}}% page 9
	\end{list}
}

% vim: set ts=3 sts=3 sw=3 tw=100 fdm=marker foldmarker=%[,%] filetype=tex spell:
