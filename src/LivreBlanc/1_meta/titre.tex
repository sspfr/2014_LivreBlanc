%[1 Page de titre
\titlehead{%
	\begin{center}%
		\ttfamily%
		\bfseries%
		\huge%
		C~o~n~f~i~d~e~n~t~i~e~l%
	\end{center}%
}

\subject{%
	Livres Bleu, vert blanc
}

\title{%
	Le contentieux\\
	FEDE --- SSP\\
	2013-2015
}

\subtitle{%
	État des lieux \& Pistes de résolutions
}

\author{%
	Gile~\bsc{Dey}%
		\footnote{Un pseudo pas secret.}
	\and
	Inco~\bsc{Gnito}%
		\footnote{Identité connue de la rédaction.}
}

\publishers{%
	À compte d'auteur
}

\date{\today}

%[1 Verso de la page de titre
\uppertitleback{%[2
	\begin{center}
		Travail proposé à~:\\SSP Fribourg\\[1em]
		Travail personnel qui n'engage pas le SSP.
	\end{center}
}

\lowertitleback{{%[2
	\small%
	\sffamily%
	\begin{center}
		Ce travail, sauf parties avec mentions contraires, est publié sous la
		licence libre
		\\[1ex]\textbf{Creative Commons-BY-NC-SA} :\\[1ex]
		\texttt{http://creativecommons.org/licenses/by-nc-sa/3.0/ch/deed.fr}
	\end{center}

	\begin{description}
		\item[BY : Paternité.]~\\
		Vous devez citer le nom de l'auteur original.
	\item[NC : Pas d’Utilisation Commerciale.]~\\
		Vous n'êtes pas autoriser à faire un usage commercial de cette Oeuvre, tout ou partie du matériel la composant.
	\item[SA : Partage des Conditions Initiales à l'Identique.]~\\
		Si vous modifiez, transformez ou adaptez cette création, vous n'avez le droit de distribuer la création qui en résulte que sous un contrat identique à celui-ci.
		En outre, à chaque réutilisation ou distribution, vous devez faire apparaître clairement aux autres les conditions contractuelles de mise à disposition de cette création.
		Chacune de ces conditions peut être levée si vous obtenez l'autorisation du titulaire des droits.
	\end{description}

	\begin{center}
		\copyright 2014 by
		G.~\bsc{Dey}
		\&
		I.~\bsc{Gnito}
	\end{center}
}}


% vim: set ts=3 sts=3 sw=3 tw=100 fdm=marker foldmarker=%[,%] filetype=tex spell:
