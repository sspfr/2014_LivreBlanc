%[1 Introduction
% Ce chapitre est constitué à partir du document fourni dans le courriel du
% Tue, 25 Nov 2014 21:33:29 +0100

\chapter{Historique}%[1
\addsec{2013}%[2
\addsubsec{Mai}%[3
\subparagraph{lundi 13}%[4
Le Conseil d'État met en consultation un paquet de mesures d'économies pour les années 2014 à 2016.

\addsubsec{Juin}%[3
\subparagraph{Mercredi 5}%[4
L'Assemblée des délégué-e-s de la FEDE vote un mandat de négociations, qui prévoit notamment:
\begin{itemize}
	\item Le refus d'entrer en matière sur les propositions du Conseil d'État.
	\item Le refus de toute mesure structurelle.
	\item L'introduction d'une clause de remboursement.
	\item Le refus de négocier au-delà de l'année 2014.
\end{itemize}

\addsubsec{Juillet}%[3
\subparagraph{Vendredi 5}%[4
Le Conseil d'État soumet une proposition d'\og ultime compromis \fg,
	qui va largement au-delà du mandat voté par l'AD de la FEDE.

\subparagraph{Mardi 9}%[4
Le Comité de la FEDE accepte d'entrer en matière sur cette proposition, en soumettant au Conseil
	d'État des adaptations mineures.

\addsubsec{Août}%[3
\subparagraph{Mardi 27}%[4
Le SSP~– Région Fribourg convoque une Assemblée générale extraordinaire pour se prononcer sur les
	mesures d'économies.
\np{130} personnes présentes participent et rejettent, à l'unanimité, les mesures d'économies.
Le Président de la FEDE était présent en début de séance, et a pu donner son point de vue.

\subparagraph{Mercredi 28}%[4
Par \np{59} voix contre \np{32}, l'Assemblée des délégué-e-s de la FEDE approuve l'accord.
Cette décision provoque un gros mécontentement au sein du personnel de l'État et du secteur
	subventionné.
Plus de \np{400} employé-e-s de l'État ont démissionné suite à cette décision.

\addsubsec{Septembre}%[3
\subparagraph{Lundi 16}%[4
Lors d'un \og Comité élargi \fg du SSP~– Région Fribourg, qui réunit plus de \np{40} délégué-e-s de
	différents secteurs,
les personnes présentes estiment qu'il se justifie d'appeler à une nouvelle mobilisation avant les
	débats au Grand Conseil (4~octobre).
Cette manifestation sera largement soutenue par: le PS, les Verts, UNIA, l'Union syndicale
	fribourgeoise (USF), la FOPIS etc.

\subparagraph{Lundi 30}%[4
\np{5} jours avant la manifestation du 4 octobre, la FEDE publie un FEDE-Infos, envoyé à tout le
	personnel de l'État, dans lequel elle indique qu'elle
		\og \emph{trouve particulièrement hasardeux de refuser ce projet pour se lancer dans un
			rapport de force avec les autorités} \fg
	et estime qu'%
		\og \emph{en s'acharnant dans cette lutte, le personnel de l'État risque des coupes sombres
		dans ses prestations et ses effectifs, sur lesquelles il n'aura plus aucune prise} \fg.

\addsubsec{Octobre}%[3
\subparagraph{Vendredi 4}%[4
La manifestation est un succès: \np{1200} employé-e-s manifestent contre les coupes salariales.

\addsubsec{Novembre}%[3
\subparagraph{Mercredi 20}%[4
AG de la FEDE, qui se déroule parfaitement bien.
Aucun reproche n'est adressé au SSP~– Région Fribourg concernant la manifestation du 4 octobre.

\addsubsec{Décembre}%[3
\subparagraph{Jeudi 19}%[4
La LDF (association des enseignant-e-s alémaniques)
demande \textbf{la suspension du SSP – Région Fribourg} de toutes les instances de la FEDE,
	en raison:
des négociations sur le travail de nuit, de notre attitude concernant les \og coupes budgétaires \fg
	et de la manifestation du 4 octobre.

\addsec{2014}%[2
\addsubsec{Janvier}%[3
\subparagraph{Jeudi 16}%[4
Le Président de la FEDE dénonce,
	dans La Liberté,
	la décision du SSP national de lui retirer son titre de secrétaire central,
	et met en cause le SSP~– Région Fribourg.
La suspension du représentant du SSP~– Région Fribourg du bureau de la FEDE est évoquée.

Le Comité du SSP~– Région Fribourg fait le choix de ne pas polémiquer par médias interposés.

\subparagraph{Mercredi 22}%[4
Le bureau de la FEDE refuse d'octroyer le droit au SSP~– Région Fribourg à un deuxième siège au
	Comité, en violation des statuts de la FEDE qui prévoient cette disposition pour les organisations
	ayant plus de \np{500} membres (article 10, alinéa 2).

\addsubsec{Février}%[3
\subparagraph{Mardi 4}%[4
Le Comité du SSP~– Région Fribourg répond à la proposition de la LDF,
	en montrant clairement que ses affirmations sont erronées,
	et que sa proposition est anti-statutaire.

\subparagraph{Mercredi 12}%[4
Le Comité de la FEDE décide de suspendre le SSP du bureau de la FEDE,
	sur la base de la demande de la LDF.

\addsubsec{Mars}%[3
\subparagraph{Lundi 17}%[4
\begin{itemize}
	\item demande à la FEDE de revenir sur la décision de suspendre le SSP du bureau de la FEDE.
	\item Accepte la création d'un groupe de travail commun.
	\item Transmet un avis de droit qui montre clairement que la décision de la FEDE est
		anti-statutaire.
\end{itemize}

\addsubsec{Mai}%[3
\subparagraph{Mercredi 7}%[4
\begin{itemize}
	\item rejette la demande du SSP~– Région Fribourg,
			sans se prononcer sur les arguments issus de l'avis de droit.
		La suspension du SSP du bureau de la FEDE est prolongée jusqu'à l'Assemblée
		des délégué-e-s du mois de novembre 2014.
	\item Préavise favorablement des propositions de modification des statuts,
		qui touchent principalement à des points relatifs au SSP~– Région Fribourg.
	\item Ne donne pas suite au projet de groupe de travail.
\end{itemize}


%[2 en plus
% \addsubsec{Juin}%[3

% \addsubsec{Juillet}%[3

% \addsubsec{Août}%[3

% \addsubsec{Septembre}%[3

% \addsubsec{Octobre}%[3

% \addsubsec{Novembre}%[3

% \addsubsec{Décembre}%[3


%[1 vim: set ts=3 sts=3 sw=3 tw=100 fdm=marker foldmarker=%[,%] filetype=tex spell:
