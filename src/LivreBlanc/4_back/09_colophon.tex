%# {{{1 Dèrnière page imprimée
%# vim: set ts=3 sts=3 sw=3 tw=100 fdm=marker spell:
%#
%# $Author: Cyril $   $Date: 2010-05-11 00:59:48 +0200 (mar, 11 mai 2010) $
%# $Revision: 437 $
%# $HeadURL: svn+ssh://sr@teke.dyndns.org/home/sr/svn.depot/formation/DAS.op/Diplome/trunk/src/ecrit/txt/meta/colophon.tex $
%#
%# }}}1

\newcommand\colophon[2]{\par\texttt{#1}\\ #2\medskip}

\cleardoublepage
\thispagestyle{empty}
\begin{center}
\vspace*{\stretch{2}}
\small
Ce document a été produit avec les outils suivant :

\begin{multicols}{2}
% \setcounter{unbalance}{3}
\par\LaTeXe{}\\ composition de document\medskip
\colophon{MetaFont}{fondeur de police et chiffres elzéviriens}
\colophon{bibtex \& \emph{jurabib}}{gestionnaire de bibliographie}
\colophon{makeidx}{gestionnaire d'index}
\colophon{KOMA-Script}{mise en page et titraille}
\colophon{Make}{gestionnaire de compilation}
\colophon{git}{getions du code source}
\colophon{Vim}{éditeur de texte}
\colophon{Archlinux}{OS permettant de gérer le tout}
\colophon{Openbox}{environement graphique}
\end{multicols}

\bigskip

Toute notre gratitude aux libristes ayant choisi de mettre leurs codes à disposition,
eux qui ont compris que le seul moyen de faire circuler le savoir est de permettre
à tout un chacun d'avoir un accès libre (free) à la connaissance.

\bigskip\bigskip

\emph{%
Le corp de texte à été volontairement porté à environ 79 caractères,
	ce qui est un bon compromis entre les standards de l'éditions (70)
	et les textes produits avec les traitements de texte amateurs.
Tout en conservant un certain confort de lecture. 
}
\vspace{\stretch{1}}
\end{center}

